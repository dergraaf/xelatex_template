% Anpassung des Seitenlayouts --------------------------------------------------
%   siehe Seitenstil.tex
% ------------------------------------------------------------------------------
\usepackage[
    automark, % Kapitelangaben in Kopfzeile automatisch erstellen
    %headsepline, % Trennlinie unter Kopfzeile
    ilines % Trennlinie linksbündig ausrichten
]{scrpage2}

\usepackage{scrhack}

% Anpassung an Landessprache ---------------------------------------------------
%\usepackage[ngerman]{babel}
%\usepackage[babel,german=quotes]{csquotes}

\usepackage[english]{babel}

% Schrift ----------------------------------------------------------------------
\usepackage{lmodern} % bessere Fonts
\usepackage{relsize} % Schriftgröße relativ festlegen

\usepackage{fontspec}

% Blindtext erzeugen "\blindtext" oder "\Blindtext"
\usepackage{blindtext}

% Grafiken ---------------------------------------------------------------------
% Einbinden von JPG-Grafiken ermöglichen
\usepackage{graphicx}
\usepackage{floatflt}

\usepackage{tikz}
\usetikzlibrary{decorations.pathmorphing,calc,shapes,shapes.geometric,patterns,snakes,matrix,arrows,positioning,automata}

% Change default placement options to [htbp]
\makeatletter
\def\fps@figure{htbp}
\makeatother

% hier liegen die Bilder des Dokuments
\graphicspath{{images/}}
%\usepackage[lofdepth,lotdepth]{subfig}
\usepackage{caption}
\usepackage[position=b]{subcaption}
%\usepackage{pdfpages}	% Einbinden von kompletten PDF Seiten
\usepackage[absolute]{textpos}

\usepackage{enumitem}

% Befehle aus AMSTeX für mathematische Symbole z.B. \boldsymbol \mathbb --------
\usepackage{amsmath,amsfonts}
\usepackage{textcomp}

\usepackage[tikz]{bclogo}

% für Index-Ausgabe mit \printindex --------------------------------------------
\usepackage{makeidx}

% Einfache Definition der Zeilenabstände und Seitenränder etc. -----------------
\usepackage{setspace}
\usepackage{geometry}

% Bilder -----------------------------------------------------------------------
% Grafiken werden nur in der dazugehörigen section eingefügt, d.h  ein zu
% spätes Einfügen in der nächsten section wird verhindert \FloatBarrier an
% Stelle, die nicht überschritten werden soll, einfügen
\usepackage[section]{placeins}


% zum Einbinden von Programmcode -----------------------------------------------
\usepackage{listings}
\usepackage{xcolor} 
\definecolor{hellgelb}{rgb}{1,1,0.9}
\definecolor{colKeys}{rgb}{0,0,1}
\definecolor{colIdentifier}{rgb}{0,0,0}
\definecolor{colComments}{rgb}{1,0,0}
\definecolor{colString}{rgb}{0,0.5,0}
\lstset{
    float=htbp,
    basicstyle=\ttfamily\color{black},%\small
    identifierstyle=\color{colIdentifier},
    keywordstyle=\color{colKeys},
    stringstyle=\color{colString},
    commentstyle=\color{colComments},
    keepspaces=true,
    columns=fullflexible,
    tabsize=4,
    captionpos=b,
%    frame=single,	%l,
    extendedchars=true,
    showspaces=false,
    showstringspaces=false,
	% zum Setzten von Referenzen auf einzelne Zeilen, Beispiel: "[[*\label{line:irq}*]]" und \ref{line:irq}
    escapeinside={[[*}{*]]},
    firstnumber=auto,
    numbers=left,
    numberstyle=\tiny,
    breaklines=true,
    backgroundcolor=\color{white},
    breakautoindent=true
}

% URL verlinken, lange URLs umbrechen etc. -------------------------------------
\usepackage{url}

% wichtig für korrekte Zitierweise ---------------------------------------------
\usepackage[square]{natbib}

% PDF-Optionen -----------------------------------------------------------------
\usepackage[
    bookmarksopen=true,
    bookmarks,
    bookmarksnumbered,
    colorlinks=true,
% diese Farbdefinitionen zeichnen Links im PDF farblich aus
    %linkcolor=red, % einfache interne Verknüpfungen
    anchorcolor=black,% Ankertext
    citecolor=blue, % Verweise auf Literaturverzeichniseinträge im Text
    filecolor=magenta, % Verknüpfungen, die lokale Dateien öffnen
    menucolor=red, % Acrobat-Menüpunkte
    urlcolor=cyan, 
% diese Farbdefinitionen sollten für den Druck verwendet werden (alles schwarz)
    linkcolor=black, % einfache interne Verknüpfungen
    %anchorcolor=black, % Ankertext
    %citecolor=black, % Verweise auf Literaturverzeichniseinträge im Text
    %filecolor=black, % Verknüpfungen, die lokale Dateien öffnen
    %menucolor=black, % Acrobat-Menüpunkte
    %urlcolor=black, 
    backref,
    plainpages=false, % zur korrekten Erstellung der Bookmarks
    pdfpagelabels, % zur korrekten Erstellung der Bookmarks
    hypertexnames=false, % zur korrekten Erstellung der Bookmarks
    linktoc=all
]{hyperref}
% Befehle, die Umlaute ausgeben, führen zu Fehlern, wenn sie hyperref als Optionen übergeben werden
\hypersetup{
    pdftitle={\documentTitle},
    pdfauthor={\documentAuthor},
    pdfcreator={\documentAuthor},
    pdfsubject={\documentTitle},
    pdfkeywords={\documentTitle},
}

% Anstelle von nomencl wird acronym weil dies einfacher ist
% - \ac{betai} fügt die Abkürzung ein. Beim ersten Vorkommen der Abkürzung wird
%   zunächst der ganze Begriff angegeben und das Akronym bzw. die Abkürzung in Klammern.
% - \acs{betai} wird dann nur das Symbol bzw. die Abkürzung im Text gedruckt.
% - \acf{betai} gibt zusätzlich noch die Erklärung aus
% - \acl{betai} fügt lediglich die Beschreibung ein.
%\usepackage[printonlyused]{acronym}

\usepackage[nonumberlist,shortcuts,acronym]{glossaries}
\makeglossaries

% Avoid grouping abbrevations according to their starting letter
\renewcommand*{\glsgroupskip}{}
% Omit the dot at the end of each description
\renewcommand*{\glspostdescription}{}
\renewcommand{\glsnamefont}[1]{\textbf{#1}}

% fortlaufendes Durchnummerieren der Fußnoten ----------------------------------
\usepackage{chngcntr}

% für lange Tabellen -----------------------------------------------------------
\usepackage{longtable}
\usepackage{array}
\usepackage{ragged2e}
\usepackage{lscape}
\usepackage{booktabs}	% schöne Tabellen
\usepackage{tabularx}

% Spaltendefinition rechtsbündig mit definierter Breite ------------------------
\newcolumntype{w}[1]{>{\raggedleft\hspace{0pt}}p{#1}}

% Show folder structures
\usepackage{dirtree}

% definiert u.a. die Befehle \todo und \listoftodos
\usepackage{todonotes}

% "\thepage\ of \pageref{LastPage}", benötigt für die Fusszeile
\usepackage{lastpage}

% sorgt dafür, dass Leerzeichen hinter parameterlosen Makros nicht als
% Makroendezeichen interpretiert werden
\usepackage{xspace}

% SI-Einheiten darstellen
\usepackage[binary-units]{siunitx}
\sisetup{
	locale=DE,
%	per=frac,
%	alsoload=binary,
%	repeatunits=false,
%	trapambigrange=false,
%	tophrase ={{ bis }},
%	decimalsymbol=comma,
}

% Use ISO date format for all dates
\usepackage[iso,english]{isodate}
%\renewcommand{\dateseparator}{-}

% for the title page
\usepackage{wallpaper}

